\documentclass{homework}

\author{Sample Student}
\class{Intro to Examples}
\title{Homework 1}
\date{\today}

\begin{document} \maketitle

% ===========================
% Question 1: Simple Proofs
% ===========================
\question Simple Theorems and Proofs
Consider the following basic examples:

\begin{theorem}
For any integer $n$, $n^2 \geq 0$.
\end{theorem}
\begin{proof}
Any integer squared is either zero or positive. 
\end{proof}

\begin{enumerate}[label=\alph*)]
	\item Even or Odd
	      \begin{theorem}
		      Every integer is either even or odd.
	      \end{theorem}
	      \begin{sol}
		      \begin{proof}
			      Let $n$ be any integer. Define:
			      \begin{itemize}
				      \item Case 1: $n = 2k$ for some $k \in \mathbb{Z}$, then $n$ is even.
				      \item Case 2: $n = 2k + 1$ for some $k \in \mathbb{Z}$, then $n$ is odd.
			      \end{itemize}
			      These cases cover all integers.
		      \end{proof}
	      \end{sol}

	\item Positive Numbers
	      \begin{theorem}
		      There exists a positive integer less than 10.
	      \end{theorem}
	      \begin{sol}
		      \begin{proof}
			      Simply choose $n = 5$, which is positive and less than 10.
		      \end{proof}
	      \end{sol}
\end{enumerate}

% ===========================
% Question 2: Spotting Errors
% ===========================
\question Identifying Mistakes
\begin{enumerate}[label=\alph*)]
	\item Consider a faulty induction attempt to prove that $n! > 0$ for $n \geq 0$. Spot the error.
	      \begin{sol}
		      The error occurs if the base case is skipped. For induction, we must verify $0! = 1 > 0$ first.
	      \end{sol}

	\item Another example: Assuming $n^2 > n$ for all $n$ without checking $n=0$.
	      \begin{sol}
		      The mistake is not considering $n=0$, for which $0^2 = 0 \not> 0$.
	      \end{sol}
\end{enumerate}

% ===========================
% Question 3: Simple State Machine
% ===========================
\question Simple State Machine
Consider a counter that halves a number until it reaches zero.

\begin{enumerate}[label=\alph*)]
	\item Starting from $n=10$, show the sequence of states.
	      \begin{sol}
		      Sequence: $10 \to 5 \to 2 \to 1 \to 0$.
	      \end{sol}

	\item Define invariant $P(n) := n \geq 0$ and verify.
	      \begin{sol}
		      Base case: $n=10 \geq 0$.
		      Step: Halving preserves non-negativity, so $P(n)$ holds for all transitions.
	      \end{sol}

	\item Explain termination.
	      \begin{sol}
		      Each transition decreases $n$, and the process stops when $n=0$. Hence, the state machine always terminates.
	      \end{sol}
\end{enumerate}

\end{document}
