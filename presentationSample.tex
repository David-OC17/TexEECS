\documentclass{presentation}

\title[Sample Theme]{Presentation Template - Example Theme}
\subtitle{A Simple Demonstration}
\author{Sample Presenter}
\institute[Sample Univ]{sample@example.edu}

\begin{document}

% ===========================
% Title Slide
% ===========================
\begin{frame}
    \titlepage
\end{frame}

% ===========================
% Outline Slide
% ===========================
\begin{frame}{Outline}
    \tableofcontents
\end{frame}

% ===========================
% Section 1
% ===========================
\section{Introduction}

\begin{frame}{Motivation}
    \begin{itemize}
        \item Introduce the main topic.
        \item Explain why it matters.
        \item Give a brief example.
    \end{itemize}
\end{frame}

\begin{frame}{Basic Equation}
    The following equation illustrates a simple relationship:
    \[
        E = mc^2
    \]
    This relates mass and energy according to Einstein’s theory.
\end{frame}

% ===========================
% Section 2
% ===========================
\section{Main Results}

\begin{frame}{Example Figure}
    \begin{figure}
        \centering
        \includegraphics[width=0.6\textwidth]{example-image}
        \caption{A sample placeholder image.}
    \end{figure}
\end{frame}

\begin{frame}[fragile]{C++ Example}
    Below is a short C++ code snippet demonstrating a simple class:

    \begin{lstlisting}
// Simple C++ example
#include <iostream>
using namespace std;

class Hello {
public:
    void greet() {
        cout << "Hello, world!" << endl;
    }
};

int main() {
    Hello h;
    h.greet();
    return 0;
}
\end{lstlisting}
\end{frame}

\begin{frame}{Summary of Results}
    \begin{enumerate}
        \item Result 1: Placeholder statement.
        \item Result 2: Placeholder statement.
        \item Result 3: Placeholder statement.
    \end{enumerate}
\end{frame}

% ===========================
% Conclusion
% ===========================
\section{Conclusion}

\begin{frame}{Key Takeaways}
    \begin{itemize}
        \item Simple example of a formatted presentation file.
        \item Shows how to structure slides with sections.
        \item Demonstrates equations, lists, figures, and code.
    \end{itemize}
\end{frame}

\begin{frame}
    \centering
    \Large Thank you!
\end{frame}

\end{document}
